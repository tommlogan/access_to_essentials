% \documentclass[]{pnas-new}
\documentclass[9pt,twocolumn,twoside,lineno]{pnas-new}
% Use the lineno option to display guide line numbers if required.
\usepackage{enumitem}


\templatetype{pnasresearcharticle} % Choose template 
% {pnasresearcharticle} = Template for a two-column research article
% {pnasmathematics} %= Template for a one-column mathematics article
% {pnasinvited} %= Template for a PNAS invited submission

\title{Making Community Resilience About People: Measuring Resilience as Access to Services} %Rethinking Resilience: Access to Essentials} %

% Use letters for affiliations, numbers to show equal authorship (if applicable) and to indicate the corresponding author
\author[]{Tom M. Logan}
\author{Seth D. Guikema}

\affil{Industrial and Operations Engineering, University of Michigan, Ann Arbor, USA}

% Please give the surname of the lead author for the running footer
\leadauthor{Logan} 

% Please add here a significance statement to explain the relevance of your work
\significancestatement{
Resilience urgently needs to be operationalized if we are going to prepare our communities for exacerbated natural hazards.
We present and demonstrate a novel framework that integrates and complements the traditional thinking on community and urban resilience that can provide actionable insight for community decision-makers.
Because essential services such as health care, education, and food are integral to everyday life, we shift the focus onto the accessibility of these services.
In devising this approach we place an emphasis on building equity, ameliorating vulnerability, and integrating with spatial planning so that our communities are empowered not only to "bounce back" from a disruption, but to "bound forward" and improve the resilience and quality of life for all residents.
}

% Please include corresponding author, author contribution and author declaration information
\authorcontributions{T.L \& S.G devised the approach. T.L collected and analyzed the data, and wrote the paper. Both edited the paper.}
\authordeclaration{The authors declare no conflict of interest.}
\correspondingauthor{\textsuperscript{*}Correspondence should be emailed to tom.logan@canterbury.ac.nz}

% Keywords are not mandatory, but authors are strongly encouraged to provide them. If provided, please include two to five keywords, separated by the pipe symbol, e.g:
\keywords{Community resilience $|$ Natural hazards $|$ Social justice $|$ Spatial planning} 

\begin{abstract}
Despite the extensive discussion surrounding resilience to climate change and natural hazards, operationalizing the concept to enable communities to build their resilience has remained challenging. 
The dominant approaches focus on either evaluating community characteristics or infrastructure functionality. 
While both remain useful, they have several limitations to their ability to provide actionable insight.
Additionally, the current conceptualizations do not consider essential services or how access is impaired by hazards. 
Given that services such as shelter, food, education, employment, and healthcare are integral to a community’s well being, access to such services is intrinsic to community resilience. 
We propose a new conceptualization of resilience that is based on access to essential services, together with a way of measuring the resilience of a community based on this conceptualization. 
Using two illustrative examples from the impacts of Hurricanes Florence and Michael, we demonstrate how decision makers and planners can use this framework to visualize and quantify interventions, that increase the resilience of our communities, in a manner that is equitable. 
\end{abstract}

\dates{This manuscript was compiled on \today}
% \doi{\url{www.pnas.org/cgi/doi/10.1073/pnas.XXXXXXXXXX}}

\begin{document}

\maketitle
\thispagestyle{firststyle}
\ifthenelse{\boolean{shortarticle}}{\ifthenelse{\boolean{singlecolumn}}{\abscontentformatted}{\abscontent}}{}


\dropcap{A}ccess to services is not something we should take for granted, not before nor after a disaster. Following Hurricane Katrina, residents of New Orleans' Lower 9\textsuperscript{th} Ward were forced to take three buses to reach their nearest grocer \cite{Netter2016-dm}. 
The 2017 South Asian floods raised fears that thousands of children permanently to dropped out of school \cite{Watt2017-bs}. 
Even without these disasters, communities throughout the world live within food deserts, health care deserts, and that list goes on. 
Access to these, and other, services is important because, without it, communities cannot function \cite{Winter1997-kc, Logan2017-fr, Dempsey2011-og}.
This makes access to essentials fundamental to community resilience.

Operationalizing resilience is among today's most impactful research questions \cite{Caldarice2019-tv}.
However, resilience as a term is conceptually malleable and multidimensional \cite{Caldarice2019-tv}. 
To capture this complexity, it is widely accepted that no single metric will be sufficient for resilience \cite{Bruneau2003-px, Sharma2018-rs, Haimes2009-gj, Levine2014-je, Cutter2014-jm, Cutter2016-landscape}.
We, as a research community, now need to develop approaches that complement one another.

% capacity metrics
One existing approach to operationalizing resilience focuses on community capacity. 
Motivating this approach is an understanding that resilience refers to qualities that enable a community to prepare for, respond to, recover from, and mitigate hazards \cite{Cutter2014-jm, Zautra2008-rb}.
Indicators are used to quantify these qualities.
These indicators capture aspects including the social, economic, institutional, and infrastructure characteristics \cite{Cutter2014-jm, Cutter2010-vg, Cutter2016-landscape, Sherrieb2010-nk}, and the vulnerability and adaptability of communities \cite{Lam2016-qn}.
This approach is not hazard-specific \cite{Koliou2018-jt}.
Rather, the objective is to determine qualities of a community that can be strengthened to enhance the community's ability to respond and recover \cite{Cutter2014-jm, Cutter2010-vg, Sherrieb2010-nk}. 

% infrastructure functionality
Infrastructure functionality is the other approach.
It focuses on critical infrastructure networks (electricity, transportation, communications, the waters) with the goal of limiting damage, mitigating the consequences, and hastening the recovery \cite{Bruneau2003-px, Barker2013-od, Hosseini2016-pm, Haimes2009-gj, Guidotti2016-vu, Curt2018-kl}.
Central to this approach is the resilience function or recovery curve, where the network's state (e.g. percent operational) is the focus.
Much of the research in this area has improved how that recovery function is quantified \cite{Bruneau2003-px, Chang2004-et, Cimellaro2010-ov, Vugrin2010-vy, Ayyub2014-mf, Sharma2018-rs}.
Other work has advanced how infrastructure networks can be optimized to reduce their vulnerability or speed their recovery \cite{Hosseini2016-pm}.
On-going advances address the interdependence of the infrastructure, to understand how failures may cascade through a system \cite{Guidotti2016-vu, Gardoni2018-xu}.
More recent extensions have begun applying the capabilities-approach \cite{Murphy2006-io}, that is, understanding how hazards affect the opportunities of individuals including being educated, being healthy etc. \cite{Gardoni2018-xu}. 
This existing work, however, remains focused on the affects from damage to critical infrastructure \cite{Guidotti2019-fc}.

% limitations
These traditional approaches are invaluable for understanding resilience. 
However, both have limitations on their ability to provide actionable insight for building resilience.
The indicators of community characteristics remain heavily focused on the social sciences \cite{Koliou2018-jt} and approaches for improvement, such as increasing the community's education, operate on decadal time-scales.
The indicators also often lack validation \cite{Bakkensen2016-ht} and are not intended to provide information regarding how a community responds to a hazard or instruction for decision-makers in those cases.
The infrastructure functionality approach, on the other hand, is useful for hazard response, but they offer little information about the community's well-being or how to enhance the resilience of residents beyond the hard infrastructure \cite{Doorn2018-fx}. 
Until very recently, the approach has ignored the community it serves and remained independent of the needs and vulnerabilities of the residents \cite{Cutter2008-placeBasedModel, Cutter2010-vg}.
Neither approach captures how a community can ensure essential services are available to all residents. 
Following a disaster, such as the aftermath of Hurricane Katrina, how many months or years must people go without acceptable access to food?

To address the unsolved challenge for community resilience, we need to integrate our understanding of the social system and the physical infrastructure and truly focus on the opportunities for residents \cite{Koliou2018-jt, Cutter2016-landscape}.
Although infrastructure is necessary for many opportunities, it, alone, is not sufficient \cite{Doorn2018-fx}.
Equally, possessing the characteristics of a strong and healthy community is vital, but it, alone, is also insufficient.

We offer a fresh perspective on community resilience.
We propose the access to essentials (ATE) resilience framework that, integrates and complements the existing approaches to provide actionable insight for communities trying to build their resilience.

\section*{Access to essentials}

\begin{figure*}
    \centering
    \includegraphics[width=\linewidth]{report/fig/NC_resilience.png}
    \caption{This is an example of the access to essentials framework can be used. This is Wilmington, NC on the 18\textsuperscript{th} of September, 2018. (a) The map of distance to nearest operational supermarket, (b) the cumulative distribution function showing the percentage of residents who are closer than $x$ to their nearest operational supermarket and service station, and (c, d) the resilience curves showing how the distribution in access changes over time.
    }
    \label{fig:fig1}
\end{figure*}

The accessibility of services such as education, healthcare, food, and recreation (the actualization of critical infrastructure) is crucial for a community’s vitality, livability, and cohesion \cite{Dempsey2011-og, Talen2003-dc, Winter1997-kc}. 
We propose an approach to measuring resilience based on access to services: the \textit{access to essentials} (ATE) resilience framework.

ATE measures the distance of residents within a community to their nearest operational essential services. 
As facilities shutter and reopen following a disruption, we can evaluate how many people are affected, the robustness, the speed of recovery, and the demographics of residents affected. 
This spatially explicit approach identifies where and who requires attention from emergency responders. 
Additionally, ATE encourages interventions to reduce service (e.g. food) deserts, both before and after a hazard, so to strengthen the community and reduce inequity.

The access to essentials resilience framework involves: 
\begin{enumerate}[topsep=1pt,itemsep=0em,partopsep=1ex,parsep=1ex]
    % \itemsep0em
    \item Engaging the community 
    \begin{enumerate}[topsep=0pt,itemsep=-2pt,partopsep=1ex,parsep=1ex]
        % \itemsep0em
        \item Establish which services are essential
    \end{enumerate}
    \item Measuring accessibility
    \begin{enumerate}[topsep=0pt,itemsep=-2pt,partopsep=1ex,parsep=1ex]
        % \itemsep0em
        \item For each of the essential services, identify the locations within the region
        \item From each block within the region, determine the network distance to each location
        \item For each block, determine the distance to the nearest operational facility
        \item Map the distances to nearest service (Figure \ref{fig:fig1}a)
        \item Plot the distribution of nearest distances (Figure \ref{fig:fig1}b)
    \end{enumerate}
    \item Monitoring the impacts from a hazard
    \begin{enumerate}[topsep=0pt,itemsep=-2pt,partopsep=1ex,parsep=1ex]
        % \itemsep0em
        \item Update the distance to nearest services as facilities open and close
        \item Construct the resilience curve showing how residents' access changes over time (Figures \ref{fig:fig1}c, S1)
        \item Intervene to build resilience (Figure \ref{fig:haz_cycle})
    \end{enumerate}
    \item Evaluating equality and equity (Figure \ref{fig:equality})
    \begin{enumerate}[topsep=0pt,itemsep=-2pt,partopsep=1ex,parsep=1ex]
        % \itemsep0em
        \item Differentiate residents based on demographics or vulnerability scores
        \item Evaluate how the access for these various groups compares
        \item Identify vulnerable areas to which to provide additional services and improve equity.
    \end{enumerate}
\end{enumerate}

%%%%%%%%%% How does this integrate the existing methods?
This is a unique way of quantifying community resilience that can support building resilience, independent of the hazard.
Computationally, ATE can be modeled using the critical infrastructure approaches to estimate the operational status of services and the community characteristics approaches to evaluate need and assess equity. 
The result is a resilience framework that integrates the existing approaches with a clear focus on the well-being of the community's residents.

\subsection*{Acceptable access}
\begin{figure}
    \centering
    \includegraphics[width=\linewidth]{report/fig/sufficient_only.png}
    \caption{The percentage of residents in each city with acceptable access, as defined by two distance thresholds, to both supermarkets and service stations. 
    }
    \label{fig:threshold}
\end{figure}

It is possible to specify a minimum acceptable standard for accessibility for each of the services and determine the portion of the community with acceptable access (Figure \ref{fig:threshold}). 
The percentage of the residents with that acceptable access is easily determined from the cumulative distribution functions (the process is described in Supplemental Text 1). 
However, not only is there no consensus on acceptable distance to most services \cite{Dempsey2008-hr}, this threshold must be place-based and service-specific and determined by engaging with the communities \cite{Pantelic1991-qu}.

We recognize that while we argue for considering access to essential services as a measure of resilience, we  currently present proximity to services. 
Access, in fact, is comprised of proximity, availability, acceptability, affordability, adequacy, and awareness \cite{Saurman2016-gj, Penchansky1981-qh}. 
Drawing on, and advancing, the relevant literature will lead to the additional dimensions being included for each service. 
These additional dimensions can be included through the use of a metric that defines acceptable access. 
This would specify a minimum level suitable for human well-being \cite{Doorn2018-fx}.
It may even require that proximity, cost, capacity, and other dimensions of accessibility vary based on the characteristics and vulnerabilities of the community to consider social justice.
This standard would be normatively indexed, i.e. the standard of acceptability is arbitrary and evolving (analogous to the poverty line, which is geographically specific) \cite{Constas2014-ui}.
This would serve as valuable extensions to the framework.

Nevertheless, proximity is a necessary component for access to services and provides insight into the resilience of a community. 
A major benefit derived from using proximity is the ability to assess the distribution of access across the population. 
There is a very real risk when using thresholds that the residents with extremely poor access, who are often among the most vulnerable, are overlooked because they are aggregated by a binary metric \cite{Logan2017-fr}. 
This is especially important given that poverty lies at the root of disaster vulnerability so true resilience approaches must help correct this \cite{Pantelic1991-qu}.

%%%%%%%%%% Equity and targeting vulnerability
\subsection*{Equality and equity}
\begin{figure}
    \centering
    \includegraphics[width=\linewidth]{report/fig/equality.png}
    \caption{Comparing how access to essentials varies between demographic groups and initially access-rich/poor residents (the top and bottom 20\% of residents). This could also be done based on indicators of social vulnerability or community capacity.}
    \label{fig:equality}
\end{figure}
Inequalities may be present before the occurrence of a hazard and are often exacerbated after an event \cite{Gardoni2018-xu}.
ATE can parse different social characteristics and evaluate the accessibility of services between demographic groups (Figure \ref{fig:equality}) (cite green space paper).
This allows for needs-based assessments and the integration with indicators of social vulnerability and community capacity.
Potential interventions can be assessed based on how they affect these different groups within the community.

%%%%%%%%%%%% Transformation
\subsection*{Promoting transformation}
“This is the United States of America. You should not have a hardship like that you have to endure;"
for years, residents of New Orleans' Lower 9\textsuperscript{th} ward had a multi-bus journey to their nearest grocery store following Hurricane Katrina \cite{Netter2016-dm}.
% Referring to the multi-bus journey to the nearest grocery store required of residents of New Orleans' Lower 9th ward for years following Hurricane Katrina  
% When discussing the motivation for opening the only grocery in the New Orleans’ Lower 9th Ward, the owner said, He was referring to the three bus journey to the nearest grocery store (How one man single-handedly opened th...). 
The many resilience approaches that prioritize "bouncing back", and quantify resilience using a "change-in functionality", risk further institutionalizing inequity \cite{Normandin2019-hp, I_Sudmeier-Rieux2014-lc, MacKinnon2013-nx}.
Claims such as "residents have grown used to” these abysmal conditions, fail to value the importance of equity and community sustainability for resilience to future events \cite{Dempsey2011-og, Pantelic1991-qu}.
They fail because they do not promote transformation that encourages communities to "bound forward."

ATE is deliberately constructed to promote transformation of communities to enhance equity, both before and after a disruption. 
This is achieved in primarily in two ways.
The first, is that unlike the critical infrastructure approach, which predominately focuses on the infrastructure damage \cite{Cutter2010-vg}, ATE assesses the utility residents derive from the system rather than its state. 
This distinction is important because restoring utility can be achieved through transformation of the state, e.g. the services can be rebuilt in different spatial configurations, rather than simplying "bouncing back."
Additionally, by assessing gross proximity, rather than the "change-in", ATE makes apparent the service-poor residents. 
Consider Figure \ref{fig:equality}, the largest change in access is experienced by the service-rich residents.
If decisions were made on the basis of this differential, then interventions would be targeted to improving the resilience of service-rich residents, and further exacerbate inequalities.
Instead, decision makers should be aware of pre- and post-hazard service deserts. 
This should mean that both mitigation and reconstruction target and improve the standard of living for all residents \cite{Pantelic1991-qu}. 
This helps to build sustainable communities, which in turn enables them to enhance their practices and adaptive capacity for future resilience \cite{Saunders2015-uz}. 

\subsection*{Integration with spatial planning}
The existing approaches to resilience are primarily spatially independent.
They do not explicitly require information about a community's layout nor do they support urban planners in making positive change. 
Integrating land use planning with resilience quantification is essential because spatial planning is among the most effective tools for reducing exposure and sensitivity to extreme events \cite{Brunetta2019-ki, Campbell2006-in, Hurlimann2012-uj} (e.g. \cite{Anderson2018-hr}). 
Surprisingly, there has been little attempt to integrate climate protection and spatial planning in practice \cite{Barnes2017-xf}.
ATE brings spatial planning to the forefront of resilience quantification by clearing linking it with urban changes and social sustainability.
This supports rethinking how our cities are designed, planned, managed, and lived in, in the pursuit of community and urban resilience \cite{Caldarice2019-tv}. 

\section*{Illustrative examples}
\begin{figure*}
    \centering
    \includegraphics[width=\linewidth]{report/fig/FL_resilience.png}
    \caption{Access in Panama City, FL on the 14\textsuperscript{th} of October, 2018. (a) The map of distance to nearest operational service station, (b) the cumulative distribution function showing how many residents are closer than $x$ to their nearest operational service station and supermarket, and (c, d) the resilience curves showing how the distribution in access changes over time.
    }
    \label{fig:fig1}
\end{figure*}
\subsection*{Overview and scope}
We now present two illustrative examples focused on Wilmington, North Carolina, and Panama City, Florida.
In late 2018 they were respectively struck by Hurricanes Florence and Michael. 
The example demonstrates how the access to two services (grocery stores and service stations) changes due to the hurricane. 
Specifically, we seek to 1) understand the spatial extent of service disruption so service-poor residents can be identified, 2) assess the resilience of the community as a result of these hazards. 
Note that our use of grocery stores and service stations is for demonstration purposes; In practice, determining the essential services as well as the desirable proximity requires stakeholder engagement within each community. 

Wilmington, NC is located on the southeastern North Carolina coast.
It has a population of approximately 120,000 people. 
Hurricane Florence made landfall slightly east of Wilmington in the early hours of September 14, 2018, as a Category 1 hurricane. 
On September 7, a week before landfall, the state governor declared a state of emergency, and on September 10 issued a mandatory evacuation order. 
Due to the hurricane’s slow movement, it resulted in heavy rainfall beginning September 13, and coupled with strong storm surge, this resulted in heavy flooding. 

By contrast, Panama City, Florida, has approximately 37,000 residents, and is located along the Emerald Coast of the Florida Panhandle. 
Hurricane Michael made landfall 40km SouthEast of Panama City as a Category 4 hurricane on October 10. 
While Florence was notable for its rainfall, 
Michael caused catastrophic damage due to extreme winds (being the strongest to hit the USA since 1992 with winds up to 208 km/h or 129 mph) and storm surge. 
A state of emergency was declared on October 7 and mandatory evacuation by the morning of October 9 was declared on October 8.

\subsection*{Inputs}
For this illustrative example we present the access to grocery stores and service stations before and following the hurricanes. 
Service locations were determined using GasBuddy\footnote{https://tracker.gasbuddy.com} and supermarkets were identified manually using Google Maps.
Access to these services was calculated at the US census block (neighborhood block) level and shapefiles and demographic data was sourced from IPUMS \cite{Manson2018-ug}. 
The Open Street Map street network was downloaded from Geofabrik\footnote{http://download.geofabrik.de/}. 
The distance from each block to all services was calculated using the Open Source Routing Machine using the approach described in Logan et. al \cite{Logan2017-fr}. 
Facility closure was recorded from GasBuddy, Twitter, and the supermarket websites.

\subsection*{Results}
Figure \ref{fig:fig1} shows the access to service stations in Wilmington, NC. The map (a) and cumulative distribution function (b) show the distance of residents to their nearest operational station on the 15th of September, one day after Hurricane Florence made landfall. Figure \ref{fig:fig1}c  shows the resilience curve of proximity to operational station, with the distribution of residents’ access shown by the bands.
Figures S3-5 show this information for supermarkets and service stations in Wilmington and Panama City. 
Note that due to data availability, the supermarket results do not include all food outlets as we only obtained information for stores that were reporting their opening times.
Although these results do not comprehensively present food-deserts, we demonstrate how we could use this approach (coupled with local connections or a mechanism for reporting closures so that all facilities are included).

The maps (a) in Figures 1, S3-5 show the distance to nearest service, for each block.
They exclude blocks with no residents.
As an example, there appears (Figure S4) to be a food-desert in southern and southeastern Panama City (based on the information we include), so these residents may require emergency food supplies even after the other stores reopen.
These maps could be varied to highlight sectors of the community with high social vulnerability, or, for example, a higher proportion of aged residents, so that emergency response can target need.

Figures \ref{fig:threshold}, S2, 1c, S3-5c show the recovery time of the services. 
Supermarkets appear to reopen faster than service stations, likely due to the resources made available by their parent companies.
In Wilmington, this was a matter of days.
Access to service stations in Wilmington was still deteriorating by the time supermarket access was almost restored (Figure \ref{fig:threshold}, S2).
This is likely due to failures in the supply chains.
However, inventory information was unavailable to us for supermarkets.

In Panama City, the recovery took significantly longer for both supermarkets and service stations.
The comparison doesn't reflect differences between the cities' resilience management due to the difference between the hurricanes, but it is clear that Panama City suffered more and for longer.

In both cities, the access to supermarkets is dismal.
Even before the hurricanes, only 30\% of residents in Wilmington live within 1 mile (1600 meters) and this is further than the majority of distance thresholds considered acceptable (e.g. \cite{Talen2003-dc}).
In Panama City this worse still, but again the access is skewed due to our omission of local food stores that would be included in practice.
Nevertheless, it demonstrates that there are likely service-deserts existing within the cities that could be mitigated prior to a hazard.

\section*{Application throughout the hazard cycle}
ATE can enhance decision making throughout the hazard management cycle. 
The cycle (Figure \ref{fig:haz_cycle}) involves preparing for and mitigating potential hazards; emergency response; and recovery, including the immediate rehabilitation and longer term (re)construction: opportunity development.

Operationalizing this framework in the field will require real-time information about the functioning of amenities and essential services.
For example, local networks or reporting systems could be implemented.
This, coupled with improvements in proximity analysis \cite{Logan2017-fr, noel2019-pypi}, mean that essential service access can be evaluated before, during, and after a hazard strikes.
This can be used to guide emergency response as well as short-term and long-term recovery and development. 

\subsection*{Mitigation and preparedness}
Before any hazard occurs, preparations should be on-going.
Existing inequities to service-deserts should be addressed, so community cohesion and social capital is enhanced \cite{Dempsey2011-og} and so residents can utilize all opportunities to improve their capacity \cite{Cutter2010-vg}.
Additionally, "what-if" analysis can help to determine which facilities are most critical in servicing the community. 
This type of analysis can be used to build redundancy or robustness into the system \cite{Wardekker2010-hw}.

\subsection*{Emergency response}
During and immediately following a disruption, ATE enables responders to identify impacted areas and allocate resources appropriately. 
Making the service accessibility map real-time would support targeting emergency supplies like shelter, food, and health care to places in need. 
Based on population characteristics, vulnerabilities and needs could be considered so that situations such as the ignoring of vulnerable residents in the Rockaways, NY, following Hurricane Sandy \cite{Subaiya2014-qx}, do not reoccur. 


\subsection*{Rehabilitation} 
During this phase, short term and basic essential services are restored \cite{Resendiz-Vazquez2019-ol}. 
Given we know which facilities are closed, optimization can be used to prioritize facility reopening to maximize accessibility. 
Longer term amenities, perhaps recreational or spiritual (although amenity importance is community and culture specific), can be prioritized for restoration in the same manner.

\subsection*{Opportunity development}
During reconstruction we should be building back better \cite{Resendiz-Vazquez2019-ol} by not only enhancing protection against future hazards \cite{Platt2019-lx}, but by improving equity and residents’ quality of life \cite{Pantelic1991-qu}. 
This is why the latter phase of recovery is referred to as “opportunity development” rather than reconstruction restoring existing conditions \cite{Resendiz-Vazquez2019-ol,Pantelic1991-qu}. 
This phase, which converges into future preparedness and mitigation, could take years and requires long-term thinking about the growth and demographic shifts of the community. 
In this phase, urban planning must be leveraged to encourage desired amenities such as grocery stores to establish in certain locations. 
For example, comprehensive plans can be used to set minimum numbers for food retailers, zoning mechanisms can simplify the regulatory process, and subsidies or other incentives can recruit retailers to in-need areas \cite{Raja2010-cm, Raja2008-wx}. 

\begin{figure}
    \centering
    \includegraphics[width=\linewidth]{report/fig/Figure_hazardCycle.pdf}
    \caption{This resilience function (aka recovery curve) shows how the access, and its distribution, may change before, during, and after a hazard. The hazard cycle shows how the ATE resilience framework can be utilized by decision-makers from mitigation to recovery.}
    \label{fig:haz_cycle}
\end{figure}

\section*{Conclusion}
The urgent need for communities to build their resilience means that operationalizing resilience is among today's most impactful research questions \cite{Caldarice2019-tv}.
While there has been significant work on resilience, the existing approaches are limited in the actionable insight they provide.
No resilience measure currently focuses on the provision of everyday amenities such as food, health care, and education, which are vital for residents to participate in life.
The access to essentials (ATE) resilience framework that we propose integrates key aspects of the traditional approaches to resilience and complements their use with the intention of maintain, restoring, and improving equitable access to essential services.

ATE provides a spatially explicit approach to quantifying resilience of access to services with a direct focus on people's well-being. 
It involves measuring the access of residents to the services and monitoring how that access changes before, during, and after a hazard event. 
Critical to our framework is the ability to discern how access changes between different demographics and vulnerable groups within a community.
Equally important is that we've devised the framework in a way that promotes continuous improvement of access to all residents and transforming the system, rather than bouncing back to pre-event conditions.
ATE has utility during all phases of the hazard cycle by providing actionable information to decision makers from preparation to post-event improvement.
By being spatially explicit, ATE integrates resilience quantification with urban planning which is crucial for our society's response to evolving threats exacerbated by climate change.

To end-users, we reiterate that while this approach is adaptable and scalable, resilience is place-based and therefore community specific, so the application of this framework must proceed community engagement and understanding.

Rethinking resilience as access to essential services promotes bounding forward, rather than bouncing back. It complements and integrates aspects of both dominant existing approaches to community resilience. By shifting the focus from the simple state of the infrastructure to the utility that the built environment provides, transformation is encouraged to improve access. This naturally enhances adaptive capacity of the community and existing capacity indicators can be used to prioritize vulnerable residents. The access to essentials framework formalizes resilience in a way that enables and encourages communities to build their resilience, equitably. 

% \matmethods{\subsection*{Data}
% We use the Centers for Disease Control and Prevention's (CDC's) 500 cities data 

% }
% \showmatmethods{} % Display the Materials and Methods section

\acknow{TL is funded by a University of Michigan Rackham PreDoctoral Fellowship and this support is gratefully acknowledged.}

\showacknow{} % Display the acknowledgments section

% Bibliography
\bibliography{references.bib}

\end{document}